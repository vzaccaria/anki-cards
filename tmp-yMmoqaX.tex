% Created 2018-04-03 Tue 13:03
% Intended LaTeX compiler: pdflatex
\documentclass[presentation]{beamer}
\usepackage[utf8]{inputenc}
\usepackage[T1]{fontenc}
\usepackage{graphicx}
\usepackage{grffile}
\usepackage{longtable}
\usepackage{wrapfig}
\usepackage{rotating}
\usepackage[normalem]{ulem}
\usepackage{amsmath}
\usepackage{textcomp}
\usepackage{amssymb}
\usepackage{capt-of}
\usepackage{hyperref}
\usetheme{default}
\author{Vittorio Zaccaria}
\date{\today}
\title{Algebra}
\hypersetup{
 pdfauthor={Vittorio Zaccaria},
 pdftitle={Algebra},
 pdfkeywords={},
 pdfsubject={},
 pdfcreator={Emacs 25.3.1 (Org mode N/A-fixup)}, 
 pdflang={English}}
\begin{document}

\maketitle
\begin{frame}{Outline}
\tableofcontents
\end{frame}


\section{Algebraic structures with one operator}
\label{sec:orgec765ff}
\subsection{Other structures}
\label{sec:org45fff7f}
\begin{frame}[label={sec:org0b38779}]{Magma}
A magma is set \(M\) equipped with a single binary operation,
\((*): M \times M \rightarrow M\) that satisfies:

\begin{itemize}
\item \alert{closure}: \(a \in M, b \in M \rightarrow a * b \in M\)
\end{itemize}
\end{frame}

\subsection{Monoids}
\label{sec:org3582f53}
\begin{frame}[label={sec:orgd749f48}]{Monoid}
A monoid \((M,*)\) is a \alert{semigroup} with a unit element (\(1 \in M\)) where the \(*\)
operation satisfies:

\begin{itemize}
\item \alert{identity}: \(m * 1 = m, \forall m \in M\).
\end{itemize}
\end{frame}

\begin{frame}[label={sec:org2e7b4ae}]{Generator set (Monoid)}
A subset \(S\) of \(M\) is said to be a generator if all elements \(m \in M\)
can be obtained by some kind of \alert{fold} of the elements \(s \in S\).

\[
      \forall m \in M, \exists s_1, \ldots, s_n \in S \textrm{~~s.t.~~} m = s_1 * \cdots * s_n
    \]
\end{frame}

\begin{frame}[label={sec:orgaa775c6}]{Free generator set (Monoid)}
A subset \(S\) of \(M\) is said to be a generator if all elements \(m \in M\)
can be obtained by a \alert{unique fold} of the elements \(s \in S\).

\[
      \forall m \in M, \exists s_1, \ldots, s_n \in S \textrm{~~s.t.~~} m = s_1 * \cdots * s_n
    \]

Not all generator sets are free. For example, \(\{1\}\) is a generator for
\(\mathbb{Z}_n\) but it is not free, in the sense that each element of
\(\mathbb{Z}_n\) can be defined by multiple sequences of 1.
\end{frame}

\begin{frame}[label={sec:org20934c5}]{Free monoid}
\begin{itemize}
\item A free monoid is a monoid \alert{with a free generator}.

\item Example: \((\mathbb{N}, +)\) is the \alert{free monoid generated by 1}.
\end{itemize}
\end{frame}

\begin{frame}[label={sec:org76d08d1}]{Abelian monoid}
A monoid \((M,*)\) where $$\forall m_1,m_2 \in M, m_1 * m_2 = m_2 * m_1$$
\end{frame}

\begin{frame}[label={sec:org7d4aa7f}]{Monoid homomorphisms}
Given two monoids \((X,\star)\) and \((Y, *)\), a homomorphism $$f: (X,\star)
    \rightarrow (Y,*)$$ is a function \(f: X \rightarrow Y\) such that:

\[
      f(1_X) = 1_Y \textrm{~~and~~} f(m \star n) = f(m) * f(n)
    \]
\end{frame}

\subsection{Groups}
\label{sec:org24b7cac}
\begin{frame}[label={sec:orgc74f5d2}]{Group}
A \href{https://en.wikipedia.org/wiki/Group\_(mathematics)}{group} \(G\) is a monoid where for all elements \(a\), the (*) operation satisfies:

\begin{itemize}
\item \alert{closure}: \(a \in G, b \in G \rightarrow a * b \in G\)
\item \alert{identity element}: \(a * 1_G = 1_G * a = a\)
\item \alert{invertibility}: \(a * a^{-1} = 1\).
\item \alert{associativity}: \(a * (b * c) = (a * b) * c\)
\end{itemize}

In a group, every element has its own inverse.
\end{frame}

\begin{frame}[label={sec:orgb369ccf}]{Subgroup}
\(H\) is a \href{https://en.wikipedia.org/wiki/Subgroup}{subgroup} of \(G\) if:

\begin{itemize}
\item \(H \subseteq G\)
\item It is closed under (*).
\end{itemize}

The \alert{trivial subgroup} of any group is the subgroup \({e}\) consisting of just the identity element.
\end{frame}

\begin{frame}[label={sec:org766cad7}]{Normal subgroup}
\(H\) is a normal subgroup of \(G\) iff $$\forall h \in H \wedge g \in G, ghg^{-1} \in H$$
\end{frame}

\begin{frame}[label={sec:org1a0bee3}]{Kernel of a group homomorphism}
The \alert{kernel} \(\textrm{ker}(f)\) of a group homomorphism \(f: (G,\star)
    \rightarrow (H,*)\) is a set \(K\) such that

\begin{itemize}
\item \(K \subseteq G\)
\item \(\forall k \in K, f(k) = 1_{*}\)
\end{itemize}
\end{frame}

\begin{frame}[label={sec:org60dacd7}]{Range of a group homomorphism}
The \alert{range} \(\textrm{range}(f)\) of a homomorphism  \(f: (G,\star)
    \rightarrow (H,*)\) is a set \(R\) such that

\begin{itemize}
\item \(R \subseteq H\)
\item \(\forall r \in R, \exists g \in G \textrm{~s.t.~} r = f(g)\)
\end{itemize}
\end{frame}

\begin{frame}[label={sec:orgf8f60e6}]{Coset}
Given a subgroup \((H, \star) \subseteq (G, \star)\), we define the \alert{left coset} of \(H\) as
$$H_g = \{g \star h : h \in H\}$$
\end{frame}

\begin{frame}[label={sec:org9e4b754}]{What is the set of cosets of a normal subgroup \(H\) of \(G\)?}
It is a partition of \((G, \star)\) and it is a group where the operation is defined as
$$Hg_1 \star Hg_2 = H(g_1 \star g_2)$$
\end{frame}


\begin{frame}[label={sec:org66cf4cc}]{Subgroup of an abelian group}
Every subgroup of an abelian group is normal.
\end{frame}

\begin{frame}[label={sec:org0a5b843}]{Coset multiplication}
Coset multiplication operation: $$Hg_1 * Hg_2 = H(g_1 \star g_2),~~g_1,g_2 \in G$$
\end{frame}

\begin{frame}[label={sec:org62731c6}]{Factor/quotient group}
\begin{itemize}
\item If \(H\) is a normal subgroup of \(G\), coset multiplication is a group
operation of a particular group: the \alert{factor (or quotient) group}.

\item The factor group \(G/H\) is a group where elements are the
cosets of \(G\) by \(H\) and group operation is cosets multiplication.
\end{itemize}
\end{frame}

\begin{frame}[label={sec:org293ac28}]{Construct a quotient group \(Z/6Z\).}
\begin{itemize}
\item \((G, +) = Z\)
\item \((H, +) \subseteq (G,+) = 6Z\) is the subgroup of all multiples of 6 in \(Z\)
\item \alert{elements} of the quotient group are all the unique cosets $$H_g = \{ 6Z + g, g \in Z \}$$
(their elements are infinite, but there are only 6 of them, \(H_1=H_7=\ldots\)).
\item the \alert{operation} among cosets $$Hg_1 + Hg_2 = H_{g_1 + g_2}$$  respects
the quotient group properties.
\end{itemize}

This quotient group is also called \alert{the cyclic group \(Z_6\)}.
\end{frame}

\begin{frame}[label={sec:org1fcbd44}]{Fundamental homomorphism theorem}
\begin{itemize}
\item Given a quotient group \(G/H\), there is a homomorphism \(\mu: G \rightarrow
      G/H\). This homomoprhism is called 'natural' and corresponds to assigning
to each \(g\) its coset \(H_g\).

\item Given any homomorphism \(\phi: G \rightarrow Q\) with its kernel \(H\), \(Q\) is
always isomorphic to \(G/H\).
\end{itemize}
\end{frame}


\begin{frame}[label={sec:org5adbbca}]{How many elements has each coset of \(H\) (a subgroup of \(G\))?}
All cosets of \(H\) have \alert{the same amount} of elements of \(H\)
\end{frame}

\begin{frame}[label={sec:orgedf17cd}]{What is the size of a group \(G\) with respect to the size of the cosets of its subgroup \(H\)?}
The size of \(G\) is a multiple of the size of \(H\) (and in turn, also of the size of its cosets).
\end{frame}

\begin{frame}[label={sec:org26868d9}]{Semigroup}
A \href{https://en.wikipedia.org/wiki/Semigroup}{semigroup} \(S\) is a "group without inverse".

\begin{itemize}
\item \alert{closure}: \(a \in G, b \in G \rightarrow a * b \in G\)
\item \alert{identity element}: \(a * 1_G = 1_G * a = a\)
\item \alert{associativity}: \((a * b) * c = a * (b * c),~~a,b,c \in G\).
\end{itemize}
\end{frame}



\begin{frame}[label={sec:org0c98aec}]{Group identity and inverse properties}
Both the identity and the inverse of every element are \alert{unique}.
\end{frame}

\begin{frame}[label={sec:orgb71c60d}]{Endomorphism ring of an abelian group}
Given a group \((G,+,0_G)\), we can define a set
\[
      End(G) = \{ f: G \rightarrow G | f(a + b) = f(a) + f(b) \}
    \]

The set \(End(G)\) can be made into a \alert{ring} with the following
operations/properties

\begin{itemize}
\item sum: \((f ++ g)(x) = f(x) + g(x)\)
\item sum unit: \(f = x \mapsto 0\)
\item sum inverse: \(f^{-1}x = (f x)^{-1}\)
\item multiplication: \((f \circ g)(x) = f(g(x))\)
\item multiplication unit: \(f = x \mapsto x\)
\end{itemize}
\end{frame}

\begin{frame}[label={sec:org342e68c}]{Order of a group element}
In a group, the least integer \(n\) such that \(a^n=1_{*}\), then that element
has order \(n\).
\end{frame}

\begin{frame}[label={sec:orgb2981c8}]{Cyclic group}
If \(G\) has an element \(a\) and all its elements are powers of \(a\), then \(G\) is cyclic:
$$G = \{ a^n : n \in Z \}$$

\(a\) is called generator and the group order is the order of \(a\).
\end{frame}

\begin{frame}[label={sec:org12cc864}]{Symmetric group}
The symmetric group \(S_n\) is the group of all permutations (symmetries) of \(\{1, . . . , n\}\)
\end{frame}

\begin{frame}[label={sec:orgb5ba748}]{Group action}
\begin{itemize}
\item A group action \(\phi\) of group \(G\) on a set \(X\) is a function
$$*: G \times X \rightarrow X$$ satisfying the following properties:

\begin{itemize}
\item identity: \(e * x = x\)
\item compatibility: \((gh) * x = g * (h * x)\)
\end{itemize}

\item An action \(g * x\) is the same as a group homomorphism \(G \rightarrow \textrm{Aut}(X)\)
\end{itemize}
\end{frame}

\begin{frame}[label={sec:orgc17273e}]{Commutator subgroup}
\begin{itemize}
\item A commutator subgroup \([G,G]\) is an abelian group that can be built from a
finite group \(G\).

\item It is composed of the set of elements $$[G,G] = \{ xyx^{-1}y^{-1}, x~y \in G \}$$

\item It is a normal subgroup
\end{itemize}
\end{frame}

\begin{frame}[label={sec:org29f3ad1}]{Commutator subgroup of \(S_n\)}
The commutator subgroup of \(S_n\) is the alternating group \(A_n\). It
is the set of all even permutations in \(S_n\). Even = even number of
transpositions in which it can be written.
\end{frame}

\begin{frame}[label={sec:org3c8e464}]{Subgroup index}
The index of a subgroup H in G is defined as the number of cosets of H in G.
\end{frame}

\begin{frame}[label={sec:orgdeacb1c}]{Group orbit}
\begin{itemize}
\item Given a group \(G\), a set \(S\) and an action \(\alpha: G \times S \rightarrow
      S\) the group orbit of each element \(s\) is $$G(s) = \{ \alpha(g, s) : g \in
      G \}$$ where \(G(s) \subset S\).
\item \(G(s)\) forms a group which is isomorphic to \(G/G_s\) (where \(G_s\) is the
isotropy group).
\end{itemize}
\end{frame}
\begin{frame}[label={sec:org70307ec}]{Group stabilizer}
\begin{itemize}
\item Given a group \(G\), a set \(S\) and an action \(\alpha: G \times S \rightarrow S\),
a stabilizier of \(s \in S\) is $$\alpha(g_s) s = s$$

\item All \(g_s\) form a group, the \alert{isotropy group} \(G_s\)
\end{itemize}
\end{frame}

\begin{frame}[label={sec:org06b3ee3}]{Group centralizer}
The centralizer of an element \(a\) of a group \(G\) is the set of elements of
\(G\) that commute with \(a\): $$C_G(a) = \{ g \vert ga = ag \}$$
\end{frame}

\begin{frame}[label={sec:orgff2603e}]{Group conjugacy classes}
\begin{itemize}
\item The conjugacy class of an element \(a \in G\) is the set of elements:
$$Cl(a) = \{ gag^{-1}: g \in G \}$$

\item The class number of G is the number of distinct (nonequivalent) conjugacy classes
\end{itemize}
\end{frame}
\subsection{Group representations}
\label{sec:org07b6a26}
\begin{frame}[label={sec:org562ec2f}]{Linear group representation}
\begin{itemize}
\item A \alert{linear group representation} of \(G\) is a group homomorphism
$$\rho : G \rightarrow Aut(V)$$

\item It represents the elements of \(G\) as \alert{symmetries of the vector space} \(V\).

\item By choosing a basis in \(V\), \(Aut(V) \simeq GL(V)\).
\end{itemize}
\end{frame}

\begin{frame}[label={sec:orgd1b9fcf}]{Properties of a linear group representation}
\begin{itemize}
\item \(\rho_g \rho_h = \rho_{gh}\)
\item \(\rho_1 = Id\)
\item \((\rho_g)^{-1} = \rho_{g^{-1}}\)
\item \(\rho_g(xv + yw) = x\rho_gv + y\rho_gw\)
\end{itemize}
\end{frame}

\begin{frame}[label={sec:orgeba58a0}]{Permutation representation}
\begin{itemize}
\item The \alert{permutation representation} is associated with a symmetric group \(G =
       S_n\) and acts on a \(V_k = k^n\).

\item Assume \(v \in V_k = \sum_j a_j e_j\) with e\(_{\text{j}}\) basis. A permutation
representation \(\rho_{\pi}: S_n \rightarrow GL(k^n)\) is such that \(\rho_s
       v = \sum a_j e_{s(j)}\).

\item Practically speaking, gives a permutation matrix for each \(g \in G\).
\end{itemize}
\end{frame}

\begin{frame}[label={sec:org904a4df}]{Character of a representation}
The \alert{character} of a representation \(\rho\) is the trace of the corresponding
matrix: \(\chi_{\rho_g} = Tr(\rho_g)\). Frobenius showed there is finitely
many irreducible representations of G and that they are completely
determined by their characters.

The character is a central or class function, it depends only on the
conjugacy class of \(g\)
\end{frame}

\begin{frame}[label={sec:orgcf8797a}]{Properties of the character of a representation}
\begin{itemize}
\item \(\chi_{\rho}(1) = dim(\rho)\)
\item \(\chi_{\rho \oplus \sigma} = \chi_{\rho} + \chi_{\sigma}\)
\item \(\chi_{\rho \otimes \sigma} = \chi_{\rho} * \chi_{\sigma}\)
\item \(\chi_{\rho^*}(g) = \chi_{\rho}(g^{-1})\)
\end{itemize}

If \(k=C\), \(\chi_{\rho}(g)=\bar{\chi}_{\rho}(g^{-1})\)
\end{frame}


\begin{frame}[label={sec:org95ea15e}]{Orthogonality relations}
Given the space of functions \(F = \{ f: G \rightarrow k \}\), define the
averaging operator $$(f_1,f_2) = \frac{1}{|G|}\sum_G f_1(g^{-1})f_2(g)$$

\begin{itemize}
\item If \((\rho, V)\) and \((\rho, W)\) are not isomorphic then
\((\chi_{\rho},\chi_{\sigma}) = 0\). If they are equivalent then
\((\chi_{\rho},\chi_{\sigma}) = 1\)

\item \(dim(V^G) = (\chi_{\rho}, \chi_{triv})\)
\end{itemize}
\end{frame}
\begin{frame}[label={sec:orgdea1151}]{Sign representation}
Given any representation of the permutation group \(\rho(S_n)\),
\(\textrm{det}(\rho(S_n))\) is the corresponding sign representation as well
and it is either 1 or -1.
\end{frame}

\begin{frame}[label={sec:orgd2c1f78}]{Faithful representation}
A \alert{faithful} representation \(\rho\) is an \alert{injective} map, i.e., different \(g\)
are represented by distinct \(\rho(g)\).
\end{frame}

\begin{frame}[label={sec:org6153e03}]{Unitary representations}
\begin{itemize}
\item Assume \(V\) is a space equipped with a hermitian dot product (on \(C_2\))
that measures the distance between vectors (i.e., an Hilbert space).

\item \((\rho,V)\) is unitary (or, a unitary operator) if it preserves the
distance of vectors (\(\langle\rho(g)v,\rho(g)w\rangle = \langle v,w
      \rangle\)).

\item If the group is finite, then one can always build an unitary operator.

\item See [\href{https://en.wikipedia.org/wiki/Unitary\_representation\#Complete\_reducibility}{complete reducibility of unitary representations}].
\end{itemize}
\end{frame}

\begin{frame}[label={sec:org7a02ba8}]{Trivial representation}
\begin{itemize}
\item A representation \(\rho: G \rightarrow GL(E_k)\) is trivial if all \(g \in G\)
map to an identity matrix.
\end{itemize}
\end{frame}

\begin{frame}[label={sec:org627d49e}]{Invariant representation}
A representation \(\rho(g)\) is \alert{invariant} if \(\forall v \in V \rho(g)v \subseteq V\)
\end{frame}

\begin{frame}[label={sec:org4c2411b}]{Intertwining operator}
\begin{itemize}
\item An intertwining operator is a functor \(\Phi_{ij}: E_i \rightarrow E_j\) that
preserves representations: $$\Phi_{ij}(\rho_i(g)v)=\rho_j(g)\Phi_{ij}(v)$$.

\item As such, it introduces a homomorphism between representations $$\rho_2
    = \Phi_{12} \rho_1$$ which comply with composition.

\item We can thus say that any group \(G\) defines a \alert{category of representations} and
we can speak of its morphisms (intertwiner) as \(Hom_G(\rho_1, \rho_2)\).
\end{itemize}
\end{frame}

\begin{frame}[label={sec:org4cf5bba}]{Equivalence of representations}
\(\rho_1, \rho_2\) are equivalent if there exists an intertwining operator \(L\)
that has an inverse and such that $$L \circ \rho_1 = \rho_2 \circ L$$.
\end{frame}

\begin{frame}[label={sec:org6dd29b5}]{Subrepresentations}
Given a subspace \(F \subseteq E\), if $$\forall f \in F, ~~\rho(G)(f) \in
     F$$ then \((\rho, F)\) is a \alert{subrepresentation} of \((\rho, E)\).
\end{frame}

\begin{frame}[label={sec:org6b4fb18}]{Trivial subrepresentation}
\begin{itemize}
\item A trivial subrepresentation \((\rho, E)\) is such that \(\rho(g)e = e\).

\item in fact, \(E^G\) is isomorphic to \(Hom_G(1,E)\), i.e., the space of constant functions
over \(E\).
\end{itemize}
\end{frame}

\begin{frame}[label={sec:org5f30b89}]{Irreducible representation}
An \alert{irreducible} representation \((\rho,V)\) has as subrepresentations only
\((\rho,0)\) and \((\rho,V)\)
\end{frame}

\begin{frame}[label={sec:org4886b4d}]{Maschke's theorem}
Given:
\begin{itemize}
\item a repr. \((\rho,V)\)
\item a subrepr \((\rho, W)\), \(W \subset V\)
\end{itemize}

then, if \(|G|\) is not a multiple of \(char(k)\), there are two representations
of G naturally associated with it:

\begin{itemize}
\item the original \((\rho, W)\) and
\item \((\rho, V/W)\)
\end{itemize}

if \(|G|\) divides \(char(k)\) then neither \(V\) nor \(W\) have a G-invariant
complement.
\end{frame}

\begin{frame}[label={sec:orgf8f5eec}]{Maschke's sum of squares}
If \(V = \bigoplus V_i\) and all \(V_i\) are irreducible representations of \(G\),
then $$|G| = \sum_i dim(V_i)^2$$.
\end{frame}

\begin{frame}[label={sec:org1b5e501}]{Full reducibility}
If \(V = W_1 \oplus \ldots W_n\) and all \((\rho, W_i)\) are irreducible then
\((\rho, V)\) is fully reducible.
\end{frame}

\begin{frame}[label={sec:org07f6316}]{Representations of Abelian groups}
\begin{itemize}
\item For an abelian group, it is possible to choose a basis to make \(\rho(g)\) diagonal.

\item For G finite, matrices are going to be block diagonal.
\end{itemize}
\end{frame}

\begin{frame}[label={sec:org6a8037e}]{One dimensional representation}
\begin{itemize}
\item A one-dimensional \(k\) representation is built above a homomorphism \(\chi: G
    \rightarrow k^{\times}\) and corresponds to a representation \(\rho_k(g) =
    \chi(g)Id_{V}\) (\(\chi(\cdot)\) is in fact a scalar).

\item The set \(\hat{G}\) of one-dimensional representations of a group \(G\) is an
abelian group and it is isomorphic to \(G/G'\) where \(G'\) is the
commutator subgroup.
\end{itemize}
\end{frame}

\begin{frame}[label={sec:org911db71}]{Representations of external direct sums}
\begin{itemize}
\item An external direct sum \(V \oplus V'\) is the set of pairs that can be built

\item \(\rho_{\oplus}(g)(v,v') = (\rho(g)v, \rho'(g)v')\)

\item Practically, if we concatenate \(v,v'\) in a single vector, \(\rho_{\oplus}\)
is a diagonal block matrix of \(\rho, \rho'\).
\end{itemize}
\end{frame}

\begin{frame}[label={sec:org99e0cf1}]{Representations of dual spaces}
Given \((\rho,V)\), one can define a representation of its dual vector space
\((\rho^*, V^*)\) such that $$\rho^*(g) f = f \circ \rho^{-1}(g)$$.
\end{frame}

\begin{frame}[label={sec:org4714696}]{Representation of quotient spaces}
Assume \(\mu: E \rightarrow E/W\) the canonical map of \(E\) into its quotient
vector space. Given a representation \((\rho, E)\), we can define
\((\rho_{E/W}, E/W)\) such that $$\rho_{E/W}(g)q = \rho(g)v' \wedge \mu v' = q, q \in E/W$$.
\end{frame}

\begin{frame}[label={sec:org36ad552}]{Representation of a tensor product space}
Given \((\rho, V)\) and \((\rho', V')\), one can define
$$(\rho_{\otimes},V \otimes V') = \rho v \otimes \rho' v'$$

If input repr. are one-dimensional, tensoring becomes multiplication.
\end{frame}

\begin{frame}[label={sec:orgf3cb413}]{Representation of \(Hom\) spaces}
Given \(Hom_C(V, V') = \{ f: V \rightarrow V' \}\) linear, one can define
a representation from representations of \(V\) and \(V'\):

$$\rho_{Hom} f = \rho' \circ f \circ \rho^{-1}$$
\end{frame}

\begin{frame}[label={sec:orgab66ae8}]{Representation of conjugate vector spaces}
Given \((\rho,V)\), \(\rho\) is a representation of the
conjugate vector space \(\bar{V}\) as well.

A conjugate representation \((\rho, \bar{V})\) is \alert{equivalent} to \((\rho, V)\)
and to its dual representation \((\rho^*, V^*)\).
\end{frame}

\begin{frame}[label={sec:orgaa06323}]{Regular representation of \(G\)}
A group algebra \(K[G]\) is a vector space of elements \(\phi\) that can be written as
$$\phi: \phi(g_1)g_1 + \phi(g_2)g_2 + \ldots$$ i.e., $$K[G]=span\{g_1, g_2,
    \ldots \}$$ The regular representation \((\rho_{K}, K[G])\) is such that
$$\rho_{K}(s)\phi = \rho_{K}(s) \phi(g_1)g_1 + \ldots = \phi(g_1)(s*g_1) +
    \ldots$$
\end{frame}

\begin{frame}[label={sec:org08ccdc6}]{Regular representation of \(G\) through Cayley}
\begin{itemize}
\item By Caley, there is an isomorphism \(\rho_C: G \rightarrow S_n\) for a group
\(G\) of order \(n\) that is given by picking a permutation \(\lambda x.gx\).

\item A regular representation is just the concatenation of a permutation
representation and the cayley isomorphism: $$\rho = \rho_{\pi} \circ
      \rho_{C}$$ In practice, it associates a permutation to each \(g \in G\).

\item For example, the regular representation of a cyclic group of order \(n\)
(\(\{1,x,x^2,\ldots, x^n\}\)) associates to each element the power of a
cyclic permutation matrix \(P\).
\end{itemize}
\end{frame}

\begin{frame}[label={sec:orgca2bcae}]{Representation of function space over \(G\)}
A function space \(F(G) = \{ \phi: G \rightarrow K \}\) can be seen as composed
of members of the group algebra (similarly to the Z-transform): $$\phi:
    \phi(g_1)g_1 + \phi(g_2)g_2 + \ldots$$ i.e., $$K[G]=span\{g_1, g_2, \ldots
    \}$$ thus one can define a representation that is equivalent to the group
algebra's one.
\end{frame}

\begin{frame}[label={sec:org173193e}]{Schur's lemma}
\begin{itemize}
\item If \((\rho, V)\) and \((\rho', V)\) are irreducible representations and there
is an intertwining operator \(L\), \alert{then} $$L \neq 0 \rightarrow L
       \textrm{~is an equivalence}$$ (no middle ground).
\end{itemize}
\end{frame}
\begin{frame}[label={sec:org61bce5e}]{Schur's lemma - corollary for \(L : V \rightarrow V\)}
For irreducible representations (\(\rho,V\)) all intertwining operators \(L: V
     \rightarrow V\) have the form \(L=\lambda I\) where \(\lambda\) is a scalar. In
fact, \(\lambda\) is the root of the characteristic polynomial (an
eigenvalue) of \(\phi\).
\end{frame}
\begin{frame}[label={sec:orge55207d}]{Schur's lemma - corollary for \(L_1, L_2\) intertwining operators}
For irreducible representations (\(\rho, V\)) and (\(\rho', V'\)) all the intertwining
operators have the form \(L_1 = \lambda L_2\).
\end{frame}
\begin{frame}[label={sec:orgc381750}]{Irreducible representations of abelian groups}
\begin{itemize}
\item All irreducible representations \((\rho, V)\) of an abelian group are such that
\(dim(V)=1\) (by Schur's lemma), i.e., they are one dimensional.
\item To each \(\rho(g)\) it corresponds a scalar \(\lambda_g\), which is called \alert{character}.
\item If \(\rho\) is not irreducible, then there exists a basis such that
\(\rho(g)\) is diagonal.
\end{itemize}
\end{frame}
\section{Algebraic structures with two operators}
\label{sec:org9df4c0d}
\subsection{Semirings}
\label{sec:org1f51d7d}
\begin{frame}[label={sec:org701fa73}]{Semiring (Rig)}
A \href{https://en.wikipedia.org/wiki/Semiring}{semiring} \(R\) consists of a set \(R\) such that:

\begin{itemize}
\item \((R, +)\) is a commutative monoid with identity = 0 (note, \alert{not a group}, it
should not have an inverse).
\item \((R, *)\) is a monoid with identity = 1
\item multiplication distributes over addition
\item multiplication by 0 gives 0 (annihilates).
\end{itemize}
\end{frame}

\subsection{Rings}
\label{sec:org1e90116}
\begin{frame}[label={sec:org43dce14}]{Ring}
A \href{https://en.wikipedia.org/wiki/Ring\_(mathematics)}{ring} \(R\) consists of a set \(R\) such that:

\begin{itemize}
\item \((R, +)\) is a \alert{commutative} \alert{group} with identity=0 (note that it should have
an inverse, so we can talk about \alert{negative} values).
\item \((R, *)\) is a monoid with identity = 1 (Semigroup)
\item multiplication distributes over addition
\item multiplication by 0 gives 0 (annihilates).
\end{itemize}
\end{frame}

\begin{frame}[label={sec:orgc01d7b3}]{Commutative ring}
A commutative ring \(R\) is commutative if \((R, *)\) is commutative.
\end{frame}

\begin{frame}[label={sec:org46c990b}]{Ring divisor of zero}
\begin{itemize}
\item \(a \neq 0\) is called divisor of zero if there is \(b \neq 0\) such that \(ab = 0\)

\item Z,Q,R and C do not have divisors of zero. This means that, if the product
of two numbers is zero, then one of them should be zero.
\end{itemize}
\end{frame}

\begin{frame}[label={sec:org8bcad5e}]{Ring cancellation property}
\begin{itemize}
\item A ring has cancellation property if and only if it doesn't have divisors
of zero. Equationally: $$ ab = ac
      \rightarrow b = c $$
\end{itemize}
\end{frame}

\begin{frame}[label={sec:org2bafc68}]{Integral domain}
\begin{itemize}
\item It is a commutative ring (with multiplication commutative) that has the
cancellation property.

\item An integral domain with non-zero characteristic \(p\) is such that \(p\) is prime.
\end{itemize}
\end{frame}

\begin{frame}[label={sec:orgcf4115f}]{Integral domains and fields}
\begin{itemize}
\item Every field is an integral domain.

\item Every \alert{finite} integral domain is a field, because you can find a
multiplicative inverse for each element of the field (this depends on the
fact that there are no zero divisors).

\item A \alert{finite integral domain} is just a \alert{finite field}
\end{itemize}
\end{frame}


\begin{frame}[label={sec:orgbb840e1}]{Quaternions}
\begin{itemize}
\item Quaternions are a subring of 2x2 matrices with unity.

\item Each quaternion \(\alpha\) has a multiplicative inverse \((1/t)\bar{\alpha}\),
where \(\bar{\alpha}\) is its conjugate and \(t\) is its norm.
\end{itemize}
\end{frame}

\begin{frame}[label={sec:orgae7c59f}]{Subring}
If \(S \subseteq R\) (\(R\) is a ring) and \(S\) is closed under sum, difference
and multiplication of \(R\), then it is a subring of \(R\).
\end{frame}

\begin{frame}[label={sec:org1a5e388}]{Ideals}
A subring \(I\) of \(R\) if

\begin{itemize}
\item it absorbs elements of \(R\) by multiplication: $$\forall s \in I, \forall r
      \in R. sr \in I$$
\item it is closed under addition.
\end{itemize}

For example, even integers (but not the odd) are an ideal of \(Z\).
\end{frame}

\begin{frame}[label={sec:org576eba3}]{Principal ideal}
A \emph{principal ideal} can be generated by taking an element \(s\) of a commutative
ring \(R\) and computing the set $$ \langle s \rangle = \{ s * r: r \in R \}$$

It can be shown that it respects the ideal properties (closed under addition
and difference and absorbs multiplication).

Every ideal in \(Z\) is principal, but there are cases where an ideal is not principal.
\end{frame}



\begin{frame}[label={sec:org2c08416}]{Ring homomorphism}
If \(R\) and \(S\) are rings, a ring homomorphism \(f: R \rightarrow S\) is a
total function such that:

\begin{itemize}
\item \(f(a + b) = f(a) + f(b)\)
\item \(f(a * b) = f(a) * f(b)\)
\item \(f(1_R) = 1_S\)
\end{itemize}
\end{frame}

\begin{frame}[label={sec:orgcb04377}]{Cosets of an ideal of a ring}
A coset of an ideal \(I_a\), just as those of subgroups, is created with the
addition operation as in group theory: $$I_a = \{ a + i | i \in I\} a \in
    R$$ We can define both addition and multiplication:

\begin{itemize}
\item \(I_a + I_b = I_{a + b}\)
\item \(I_a * I_b = I_{a * b}\)
\end{itemize}
\end{frame}

\begin{frame}[label={sec:orgd58dfd4}]{Quotient ring}
Given a ring \(R\) and an ideal \(I\), the set of cosets of \(I\) is a ring and
it is called the quotient ring \(R/I\).
\end{frame}

\begin{frame}[label={sec:org7875114}]{Division ring}
\begin{itemize}
\item A ring where every nonzero element a has a multiplicative inverse.
\item Also called skew field.
\end{itemize}
\end{frame}
\begin{frame}[label={sec:org81d4ae7}]{Integral system}
\begin{itemize}
\item An integral domain \(A\) that is \alert{ordered} and for which every subset \(B \subseteq
      A\) has a \alert{least element} (initial object)
\item The above property ensures that there are no elements \(c\) such that \(0<c<1\).
\item Every element is a multiple of 1.
\end{itemize}
\end{frame}
\begin{frame}[label={sec:orgbb25b85}]{Integral system - mathematical induction}
If the following condition hold for a subset \(K\) of an ordered integral system:
\begin{itemize}
\item \(1 \in K\)
\item \(k \in K \Rightarrow (k+1) \in K\)
\end{itemize}
then \(K\) is all the integers. Proof by contradiction (see Pinter).
\end{frame}

\begin{frame}[label={sec:org9d2c62d}]{Polynomial}
\begin{itemize}
\item Take a ring \(Z\) and add a new symbol \(\pi\). Which other numbers should be
present to make it a ring? It turns out that all the numbers of the form
$$a_n \pi^n + \dots + a_1 \pi + a_0$$ should be present.
\item The ring created is the polynomial ring \(\mathbb{Z}[x]\).
\end{itemize}
\end{frame}
\subsection{Fields}
\label{sec:org550be7a}

\begin{frame}[label={sec:orgaf0b771}]{Field}
A \href{https://en.wikipedia.org/wiki/Field\_(mathematics)}{field} a set with two operations:

\begin{itemize}
\item \((F, +)\) is a commutative group
\item \((F, *)\) is a commutative group
\end{itemize}
\end{frame}

\begin{frame}[label={sec:org90b3db0}]{Field characteristic}
A field has characteristic \(n\) if \(1_{*}\) summed with it self \(n\) times gives
\(0_{+}\).

The following relation is found to hold: $$ n > 0 \leftrightarrow
    \textrm{prime}(n)$$
\end{frame}

\begin{frame}[label={sec:org8b02cfd}]{Prime field F}
A finite field of order \(p\) where \(p\) is prime.
\end{frame}

\begin{frame}[label={sec:org4ab092a}]{Finite field \(F_p\)}
It has \(p\) elements. If \(p\) is prime then \(F_p\) is a prime field and
operations are understood as modular.
\end{frame}

\begin{frame}[label={sec:orgc47181d}]{Algebraically closed field}
An algebraically closed field \(F\) contains a root for every non-constant
polynomial in \(F[x]\), the ring of polynomials in the variable \(x\) with
coefficients in \(F\).
\end{frame}
\section{Algebras of vector spaces and modules}
\label{sec:orga19d0bc}
\subsection{Vector spaces}
\label{sec:orgb9bb799}

\begin{frame}[label={sec:org86dc961}]{Vector space}
A pair \((V,K)\) where

\begin{itemize}
\item \((K,(+,0),(\cdot,1))\) is a field
\item \((V,+,e)\) is an abelian group under addition
\end{itemize}

Moreover, the following operation should be total (scalar multiplication)
\[
    *: K \times V \rightarrow V
    \]
\end{frame}

\begin{frame}[label={sec:orge557e44}]{Simple vectors}
They are ordered sequences of elements that belong to a field (\href{https://en.wikipedia.org/wiki/Scalar\_(mathematics)}{scalars)}.
Classical results of geometry apply.
\end{frame}

\begin{frame}[label={sec:orgdd11a05}]{Direct sum}
The direct sum of two vector spaces is a linear combination (span) of their basis.

$$V_1=span_k([0,1]), V_2=span_k([1,0]), V_1 \oplus V_2 = span_k([0,1],[1,0])$$
\end{frame}

\begin{frame}[label={sec:orge50fa71}]{Normed vector spaces}
A \alert{normed} vector space \(V\) is endowed with a map \(V \rightarrow R\).
\end{frame}

\begin{frame}[label={sec:org166a617}]{Inner product spaces}
An \alert{inner product} space \(V\) is endowed with an operation \(V \times V \rightarrow R\).

Hermitian products are one of the possible operations
\end{frame}

\begin{frame}[label={sec:org6caaff0}]{Dual Space}
\begin{itemize}
\item Given \(V\), one can define its dual space as the space of linear mappings
over it, i.e., \(V^* = \{f: V \rightarrow C \}\).

\item This space has basis \(f_i^*(v_j)=\delta_{i,j}\) (kronecker deltas), i.e.,
\(V^*=span(f_1^*, \ldots, f_n^*)\).
\end{itemize}
\end{frame}




\begin{frame}[label={sec:org7d7dc0d}]{Tensor product spaces}
Given two vector spaces \((V,V')\), I can define their tensor product \(W =V
    \otimes V'\) by

\begin{itemize}
\item creating monomials (concatenations) of their bases \(e_i \otimes e'_j\) as
a new basis for \(W\)

\item expressing the product \(v \otimes v' = \sum_{c_{ij}}c_{ij} e_i \otimes e'_j\) such
that the properties of a bilinear product hold.
\end{itemize}
\end{frame}

\begin{frame}[label={sec:org59807e8}]{Conjugate vector spaces}
Given a vector space on complex numbers \(V_C\) then also \(\bar{V}_C\), where
 scalar multiplication is \(\bar{z}*v\) for \(z \in C\), is a vector space.
\end{frame}

\begin{frame}[label={sec:org432a074}]{Algebra}
\begin{itemize}
\item It is a module/vector space equipped with a bi-linear operator $$ \times: V \times V
      \rightarrow V$$ acting as \alert{multiplication} between vectors.

\item The following axioms should hold:

\begin{itemize}
\item Right distributivity: \((x + y) \times z = x \times z + y \times z\)
\item Left distributivity: \(x \times (y + z) = x \times y + x \times z\)
\item Compatibility with scalars: \((ax) \times (by) = (ab) (x \times y)\).
\end{itemize}
\end{itemize}
\end{frame}

\begin{frame}[label={sec:org4b8c6de}]{Affine space}
\begin{itemize}
\item A tuple \((A, V_k, (+): V_k \times A \rightarrow A)\) defines an \emph{affine space}
\(A\) if \((+)\) complies with the group action properties.
\item Alternatively, it can be defined as \((A, V_k, (-): A \times A \rightarrow V_k)\) where
\((-)\) complies with Weyl's axioms
\end{itemize}
\end{frame}

\begin{frame}[label={sec:org261e562}]{Quotient space}
Given a vector space \(V\) and a subspace \(W\), the quotient space \(V/W\) is the set of
cosets \(W_v = \{ v + w | w \in W \}\).  Fixed a \(v\), the coset represents the elements \(y\)
for which \(\exists w \in W, x - y = w\) which is an equivalence class.
\end{frame}

\begin{frame}[label={sec:orgfa80986}]{Group Algebra as formal sum}
A group algebra \(K[G]\) for a group \(G\) and a field \(K\) is a vector space
that is the direct sum of copies of \(K\) indexed by \(G\).

In practice, the basis \(e_g\) is a vector that has a dimension \(|G|\) and
whose \(g-th\) component is 1.

Each element of the algebra is thus a formal linear combination of the
elements of \(G\) with coefficients in \(K\), i.e., $$f = \sum_{g\in G}a_{g,f}
    e_g$$ where \(e_g\) are basis in \(K[G]\).
\end{frame}


\begin{frame}[label={sec:org860b1bb}]{Group Algebra as function space}
Given a group \(G\) and a field \(K\), we can build an algebra \(K[G]\) where

\begin{itemize}
\item vectors are functions \(f: G \rightarrow K\) for \(K\) either a ring or
field.
\item \((f_1 + f_2) = g \mapsto f_1(g) + f_2(g)\)
\item \((\alpha f) = g \mapsto \alpha f(g)\)
\item \((f_1 \times f_2) = g \mapsto \sum_{u \in G}f_1(u)f_2(u^{-1}g)\)
\end{itemize}
\end{frame}



\begin{frame}[label={sec:org39c37a8}]{R-Module}
Generalization of vector space. A pair \((V,K)\) where

\begin{itemize}
\item \((K,(+ ,0),(\cdot,1))\) is a \alert{ring}
\item \((V,+,e)\) is an abelian group under addition
\end{itemize}

Moreover, the following operation should be total (scalar multiplication)
\[
    *: K \times V \rightarrow V
    \]
\end{frame}
\begin{frame}[label={sec:orge86ef2b}]{Modules over division rings}
\begin{itemize}
\item Much of the linear algebra results can be applied to modules over division
rings as the latter are almost fields.
\end{itemize}
\end{frame}

\begin{frame}[label={sec:org235637d}]{Projector}
\begin{itemize}
\item A \alert{projector} is a linear operator \(P: X \rightarrow X\) for which \(P^2=P\).

\item Both \(P\) and \((I-P)\) define a direct sum composition $$X \simeq X_1 \oplus X_2 = PX \oplus (I-P)X$$

\item One typically says that \alert{the projector \(P\) projects X the invariant image \(X_1\) with kernel \(X_2\)}
\end{itemize}
\end{frame}

\begin{frame}[label={sec:org55e3ee5}]{Projector image}
The image of a projector \(P\) is an \alert{invariant} under \(P\): $$ P(PX)=(PX) \sim Pw=w$$
\end{frame}


\begin{frame}[label={sec:orga489e38}]{Projector kernel}
Given a projector \(P\), \(X_2 = (1-P)X\) is the kernel of \(P\) by definition: $$PX_2 = P(1-P)X = (P-P^2)X = 0$$
\end{frame}




\section{Category theory basics}
\label{sec:orgdfa8141}

\subsection{Category basics}
\label{sec:org3818826}
\begin{frame}[label={sec:orga329c8c}]{Category}
A category is a triple \(\mathcal{C}(O, M, \bullet)\) that abides these laws:

\begin{itemize}
\item \alert{identity}: \(\forall o \in O, \exists id_o \in M\)

\item \alert{composition}: '\(\bullet\)' composes morphisms in \(M\) that share source and target :

\begin{enumerate}
\item \(\bullet(A \rightarrow B, B \rightarrow C) = A \rightarrow C\)

\item \((f \bullet g) \bullet h == f \bullet (h \bullet g)\)

\item \(id_x \bullet f = f \bullet id_y\)
\end{enumerate}
\end{itemize}
\end{frame}

\subsection{Monoid (Category)}
\label{sec:orge3c14e8}

\begin{frame}[label={sec:org1cde801}]{Monoid as a category}
\begin{itemize}
\item A monoid \(\mathcal{M}(M, id_0, \star)\) is just a category \(\mathcal{C}(O, M, \star)\) where O = \(\{ o_1 \}\) and \(id_0 = id_{o1}\).

\item The elements \(M\) of a monoid are the morphisms \(M\) of the corresponding
category. As such, \alert{associativity} holds.
\end{itemize}
\end{frame}

\begin{frame}[label={sec:org7ee1ba1}]{Free Monoid}
\begin{itemize}
\item A free monoid of M is just a monoid \(\mathcal{M}(List[G], [], ++)\)

\item It has a free generator set G and all its elements are uniquely determined by
a fold of elements in \(G\).
\end{itemize}
\end{frame}

\begin{frame}[label={sec:org242fae4}]{Action}
An action of a \(\mathcal{M}(M, id_0, \star)\) over a set \(S\) of states is a
function  \[ \alpha: M \times S \rightarrow S \]. The following properties should be satisfied:

\begin{itemize}
\item identity: \(\alpha(id_0, s) = s\)

\item compatibility: \(\alpha(f \star g, s) = \alpha(f, \alpha(g, s))\)
\end{itemize}
\end{frame}


\subsection{Preorder, partial and linear order (Category)}
\label{sec:orgb381835}

\begin{frame}[label={sec:org1c5bae1}]{Preorder as a category}
A \alert{preorder} is a category \(\mathcal{C}(O, M, \bullet)\), where there is at most one
morphism between objects. It has the following properties:

\begin{itemize}
\item \alert{reflexivity}: from identity morphisms

\item \alert{transitivity}: from composition of morphisms
\end{itemize}
\end{frame}

\begin{frame}[label={sec:org2288448}]{Partial order as a category}
A \alert{partial order} (\alert{poset}) is a preorder where there cannot be loops except for
identity arrows.

\begin{itemize}
\item \alert{reflexivity}: from identity morphisms

\item \alert{transitivity}: from composition of morphisms

\item \alert{antisimmetry}: \(x \rightarrow y \rightarrow x \Rightarrow x = y\)
\end{itemize}
\end{frame}

\begin{frame}[label={sec:org9ff79d0}]{Linear order as a category}
\begin{itemize}
\item A Linear order is a partial order where:

\[x \rightarrow y \in M \Rightarrow y \rightarrow x \notin M\]

\item Basically, either one or the other but all objects pairs have morphisms
\end{itemize}
\end{frame}

\begin{frame}[label={sec:org5cc972a}]{Meet}
A \alert{meet}

\begin{itemize}
\item is the unique \alert{product} of two objects in a poset

\item it is regarded as the \alert{minimum} of two objects

\item in boolean algebra, it can be seen as the \alert{and} operation
\end{itemize}
\end{frame}

\begin{frame}[label={sec:orge49673c}]{Join}
A \alert{join}

\begin{itemize}
\item is the unique \alert{coproduct} of two objects in a poset

\item it is regarded as the maximum of two objects

\item in boolean algebra, it can be seen as the \alert{or} operation
\end{itemize}
\end{frame}

\begin{frame}[label={sec:orge220707}]{Lattice as a category}
\href{https://ncatlab.org/nlab/show/lattice}{A lattice} is a poset where \alert{all objects} have a \alert{meet} and a \alert{join}
\end{frame}

\subsection{Special categories}
\label{sec:org1f87c90}

\begin{frame}[label={sec:org865e0f4}]{Categorical nomenclature (\alert{small} and \alert{large} cats)}
\begin{itemize}
\item A \alert{class} is a collection of sets that share a property.

\item Large category \(C\): Either \(ob(C)\) or both \(ob(C)\) and \(hom(C)\) are proper
classes (i.e., a class that is not a set).

\item Locally small category \(C\): \(hom(C)\) is a set (\alert{Set} is just an example)

\item Small category \(C\): \(ob(C)\) and \(hom(C)\) are sets
\end{itemize}
\end{frame}

\begin{frame}[label={sec:org1c061cc}]{Groupoid}
A (small) \alert{groupoid} is a (small) category in which all morphisms are
\alert{isomorphisms}. I.e., composition has a \alert{two sided inverse}.
\end{frame}

\begin{frame}[label={sec:org690b9ec}]{Big category}
A category of categories (\(CAT\)) that:

\begin{itemize}
\item has functors as morphisms
\item excludes itself.
\end{itemize}
\end{frame}

\begin{frame}[label={sec:org9cfa68d}]{Product category}
Given two categories \(C\) and \(D\), a product category \(C \times D\) is such that

\begin{itemize}
\item Objects are all possible pairs of original objects

\item Morphisms are all the corresponding morphisms
\end{itemize}
\end{frame}

\begin{frame}[label={sec:org6adf220}]{Cartesian category}
It is a category where a product is defined for all objects. This product is a
bifunctor \(C \times C \rightarrow C\). It is a superset of monoidal categories,
where also initial objects should be defined.
\end{frame}

\begin{frame}[label={sec:org326b848}]{Cartesian closed category}
It is a cartesian category that has exponentials and a terminal object.
\end{frame}

\begin{frame}[label={sec:orgb35d5fd}]{BiCartesian closed category}
It is a cartesian category that has exponentials and a terminal object and coproducts.
\end{frame}

\subsection{Set category}
\label{sec:orgd1e3d83}
\begin{frame}[label={sec:org9051813}]{Monomorphisms (Sets)}
\begin{itemize}
\item \(f\) is a monomorphism if $$\neg\exists (g_1, g_2) ~~ g_1 \neq g_2 \wedge f \circ g_1 = f \circ g_2$$

\item Injective functions among sets can be classified as \emph{monomorphisms};

\item \alert{Assume} three sets \(A,B\) and \(C\) and \(f: A \rightarrow B\), and \(g_1, g_2 : C \rightarrow A\).
If \(f\) is non-injective, then the pre-composition with g\(_{\text{1}}\) and g\(_{\text{2}}\) (where g\(_{\text{1}}\) and g\(_{\text{2}}\)
differ only because they map the same element \(z \in C\) into two different \(a_1,
  a_2\) for which \(f(a_1) = f(a_2)\)) will be the same: $$f \circ g_1 = f \circ g_2$$ even if those are different.
\end{itemize}
\end{frame}

\begin{frame}[label={sec:orgf6ca006}]{Epimorphisms (Sets)}
\begin{itemize}
\item \(f\) is an epimorphism if $$\neg\exists (g_1, g_2) ~~ g_1 \neq g_2 \wedge g_1 \circ f = g_2
  \circ f$$.

\item Surjective functions among sets can be classified as \emph{epimorphisms};

\item \alert{Assume} three sets \(A,B\) and \(C\) and \(f: A \rightarrow B\), and \(g_1, g_2: B \rightarrow C\).
If \(f\) is not surjective, there are elements in \(B\) which will not participate
to \(g \circ f\) (\emph{terra incognita}). There will be thus \(g_1\) and \(g_2\) that differ only in
terms of those excluded terms while their composition is the same.
\end{itemize}
\end{frame}

\begin{frame}[label={sec:orgae1335a}]{Terminal object (Sets)}
There is a set 1 for which, for any set \(X\), there is a unique function \(X \rightarrow 1\).
This is called the \alert{terminal object}.
\end{frame}

\begin{frame}[label={sec:orgcfe1229}]{Unit of categorical product (Sets)}
The unit of a categorical product is the terminal object, \(X \times 1 \simeq X\)
\end{frame}

\begin{frame}[label={sec:org844b998}]{Sets sharing an element (Sets)}
If:

\begin{itemize}
\item there is a monomorphism \(m: B \rightarrow X\)

\item and there is \(k: 1 \rightarrow B\) such that \(x: 1 \rightarrow X\) factors through \(m\), i.e.,  \(x = m \circ k\)
\end{itemize}

then \(x \in B\)
\end{frame}

\begin{frame}[label={sec:org52da7bb}]{Subobject (Sets)}
Any object \(B\) for which there exists a monomorphism \(B \rightarrow X\) is a subset/subobject of \(X\).
\end{frame}

\begin{frame}[label={sec:orgdc37ba3}]{Equalizer (Sets)}
Given two functions (\(g_1, g_2: X \rightarrow Y\)), an equaliser is an \alert{object} and
\alert{monomorphism} \alert{pair} \((E,m: E \rightarrow X)\) for which the following
properties hold:

\begin{enumerate}
\item \alert{Equivalence}: \(g_1 \circ m = g_2 \circ m\)
\item \alert{Limit}: for any other object pair \((O,m_o: O \rightarrow X)\) where \(g_1 \circ
   m_o = g_2 \circ m_o\), there exists a unique morphism \(f: O \rightarrow E\) such
that \(m_o = m \circ f\)
\end{enumerate}

\(E\) should be understood as the subset of elements of \(X\) for which \(g_1(x) =
g_2(x)\), i.e., the solutions of the equation.
\end{frame}

\begin{frame}[label={sec:orgaf0ec9b}]{Function objects (Sets)}
\begin{itemize}
\item We can use a \alert{universal construction} for a function object \(f\) which represents the
set of functions from object \(a\) to \(b\); we call this object \(b^a\).
\item Ideally, \(f\) is such that there exists a morphism \(g: f \times a
      \rightarrow b\) and for all other morphisms \(g': (f' \times a) \rightarrow b\)
there is a unique morphism \((h \times Id): (f' \times a) \rightarrow (f
      \times a)\) such that \(g' = g \circ (h \times Id)\).
\end{itemize}
\end{frame}

\begin{frame}[label={sec:orge01450f}]{Currying}
\begin{itemize}
\item Assume \(b^a\) the function object from \(a\) to \(b\)
\item If \(f\) is a function it can be seen as a subobject (member) of \(b^a\). As
such, by universal construction of exponentials, there is an one to one
correspondence called \alert{currying}: $$f \rightarrow (b^a) \sim (f \times a) \rightarrow b$$
\end{itemize}
\end{frame}

\subsection{Kleisly category}
\label{sec:org1e04b8c}
\begin{frame}[label={sec:orgc26e6a5}]{Kleisly category (\(C_T\)) definition}
\begin{itemize}
\item Assume \(C\) is a category with an endofunctor \(T\) and a morphism \(\mu: T^2 C \rightarrow C\)
\item \(C_T\) has the same object as \(C\) but any morphism \(A \rightarrow_T B\) is
built by picking a morphism \(A \rightarrow T B\) in the following way:
\begin{itemize}
\item The \alert{identity} for any \(A\) is constructed by picking a morphism \(\eta_A: A
        \rightarrow T A\)
\item The composed arrow \(h_T = f_T \circ g_T : A \rightarrow_T C\) is the
the one built as \(h: A \rightarrow T C\) such that $$ h = \mu
        \circ T f \circ g$$
\item Note that \(T f: T B \rightarrow T^2 C\)
\end{itemize}
\end{itemize}
\end{frame}


\subsection{Monoidal category}
\label{sec:org6b6bc0b}
\begin{frame}[label={sec:orgaec9b92}]{Monoidal category definition}
\begin{itemize}
\item A category \(C\) is \alert{monoidal} if \alert{all} pairs of objects have a product
(co-product) and a terminal (initial) object.

\item Both product and coproduct in a monoidal category are individuated by the
same symbol, i.e., \alert{tensor product} (\(\otimes\))
\end{itemize}
\end{frame}

\section{Functors}
\label{sec:org7d5e3ac}
\subsection{Introduction}
\label{sec:orgb1792e3}
\begin{frame}[label={sec:orgfd350a7}]{Functor definition}
A functor \[ F : \mathcal{C} \rightarrow \mathcal{C'} \] is a pair \((F_o, F_m)\) where

\begin{itemize}
\item \(F_o\) maps objects across categories \(\mathcal{C}\) and \(\mathcal{C'}\), while

\item \(F_m\) maps morphisms with laws \(F(id_o) = id_F(o)\) and \(F(h \bullet g) = F(h) \bullet F(g)\)

\item It preserves composition and identity
\end{itemize}
\end{frame}

\begin{frame}[label={sec:orgb09facb}]{Full functor}
A \alert{full functor} \(T: C \rightarrow D\) is an \alert{epimorphism} between morphisms in \(C\) and \(D\).
\end{frame}

\begin{frame}[label={sec:org4bbff7d}]{Faithful functor}
A faithful functor \(T: C \rightarrow D\) is a \alert{monomorphism} between morphisms in \(C\) and \(D\).
\end{frame}


\begin{frame}[label={sec:org7453bd5}]{Identity functor}
\(Id: C \rightarrow C\) is a functor that maps an object to itself and a
function to itself.
\end{frame}

\begin{frame}[label={sec:orgc8759d7}]{Constant \(\Delta_c\) functor}
A constant functor \(\Delta_c: B \rightarrow C\) is a functor that maps every
object in \(B\) into a single object \(c \in C\).
\end{frame}

\begin{frame}[label={sec:orgd743c18}]{Bifunctors}
\begin{itemize}
\item A bifunctor over \(C\) and \(D\) is a functor over pairs of objects and morphisms,
i.e., \(C \times D \rightarrow E\)

\item Products and co-products are special bifunctors \(C \times C \rightarrow C\)
\end{itemize}
\end{frame}

\begin{frame}[label={sec:orgb688f43}]{Contravariant functors}
A contravariant functor is a functor \(C^{op} \rightarrow D\).
\end{frame}

\begin{frame}[label={sec:org8eaa817}]{Profunctors}
A pro-functor is a functor \(C^{op} \times C \rightarrow C\). It can be used
to describe the functoriality of the 'arrow' (exponential) object construction.
\end{frame}

\begin{frame}[label={sec:orgc163af0}]{Combination of functors}
\begin{itemize}
\item \(T(X) = X = Id\) is a functor

\item \(T(X) = A = \Delta_A\) is a functor

\item If \(F_1(X)\) and \(F_2(X)\) are functors then \(T(X) = F_1(X) + F_2(X)\) is a
functor
\item If \(F_1(X)\) and \(F_2(X)\) are functors then \(T(X) = F_1(X) * F_2(X)\) is a functor

\item If \(F_1(X)\) is a functor then \(T(X) = F_1(X)^A\) is a functor
\end{itemize}


Thus any polynomial expression in an object \(X\) can be made into a functor.
\end{frame}

\begin{frame}[label={sec:org788b71a}]{Curry-Howard-Lambek isomorphism}
\begin{center}
\begin{tabular}{llllll}
\alert{Logic} & \(\top\) & \(\bot\) & \(a \wedge b\) & \(a \vee b\) & \(a \Rightarrow b\)\\
\alert{Types} & () & Void & (a,b) & Either a b & \(a \rightarrow b\)\\
\alert{C. C. Category} & terminal & initial & \(a \times b\) & \(a + b\) & \(b^a\)\\
\end{tabular}
\end{center}

\begin{itemize}
\item Proving a logic predicate means constructing an element of a specific type

\item A cartesian closed category is a model for logic and category theory
\end{itemize}
\end{frame}
\begin{frame}[label={sec:orgaccad44}]{Natural transformation of functors \(F \rightarrow G\)}
It is a way of comparing functors; given two functors \(F,G: C \rightarrow D\), I
can create compare them by :
\begin{itemize}
\item picking to objects object \(c_1,c_2\) and a morphism \(f: c_1 \rightarrow c_2\)
\item picking in \(D\) a morphism \(\alpha_{c_1}\) in \(D\) that maps \(Fc_1\) to \(Gc_1\)
(the family of \(\alpha_c\) is called \alert{components of the NT}).
\item picking in \(D\) a morphism \(\alpha_{c_2}\) in \(D\) that maps \(Fc_2\) to \(Gc_2\)
\end{itemize}
How can I detect any relation between \(F,G\) meaning they are similar? This
naturality condition is $$\alpha_{c_2} \circ F f = G f \circ \alpha_{c_1},
    \forall c_1,c_2$$
\end{frame}

\begin{frame}[label={sec:org9258177}]{Natural transformation maps morphisms to \ldots{}}
A natural transformation maps a morphism to a commuting diagram (\alert{naturality
square}).
\end{frame}

\begin{frame}[label={sec:org5e71652}]{Natural isomorphims}
All the components of a NT are invertible
\end{frame}



\section{Limits}
\label{sec:org3b2836d}
\subsection{Basic limits theory}
\label{sec:org752afb4}
\begin{frame}[label={sec:orga82c59c}]{Selecting objects}
In any category, I can pick objects by expressing a pattern. This defined
through another another category and a functor.

Picking two objects from \(C\) means a functor \(\textbf{2} \rightarrow C\).
\end{frame}

\begin{frame}[label={sec:orgdc86793}]{Products as limit}
\end{frame}
\section{Functor algebra}
\label{sec:org8ecca67}
\subsection{Functor algebra introduction}
\label{sec:orgdbf39b5}
\begin{frame}[label={sec:org36c2895}]{Definition of algebras and co-algebras}
\begin{itemize}
\item An algebraic operation in domain \(X\) is a morphism from tuple objects to \(X\)
e.g.: $$ X \times A
      \rightarrow X$$ or $$ A \rightarrow X $$

\item A co-algebraic operation in domain \(X\) is a morphism from \(X\) into a tuple
object, possibly containing \(X\): $$ X \rightarrow A \times
      X $$
\end{itemize}
\end{frame}

\begin{frame}[label={sec:org4f065c5}]{Functoriality of products, coproducts, exponential and powersets}
\begin{itemize}
\item The product operation in a category \(C\) is a bi-functor \((-) \times (-): C
      \times C \rightarrow C\) (it maps both \alert{pairs of objects} to a \alert{product object}
and \alert{pairs of functions} to a \alert{product function}).

\item Fixed an object \(A\) there is a functor \((-)^A: C \rightarrow C\) that maps \(X\)
into \(X^A\) (the exponential object).

\item The identity map is (endo-)functorial as well as the endomap that maps \(X\)
into a fixed object.
\end{itemize}
\end{frame}

\begin{frame}[label={sec:orgf859fc8}]{Functoriality when combining of product and coproducts}
\begin{itemize}
\item If we can build functors for products and coproducts, we can build functors
also for their combination;

\item For example: \(T(X) = X + (C \times X)\)
maps X to an object \(X + (C \times X)\) but also morphisms \(f: X \rightarrow X'\):

$$T(f) = f + (id_C \times f): X+(C\times X) \rightarrow X'+(C \times X')$$
\end{itemize}
\end{frame}

\begin{frame}[label={sec:orgcb2cd62}]{Functoriality for initial and final object isomorphisms}
\begin{itemize}
\item \(X \rightarrow 1\) is the functor that maps \(X\) into the singleton set (final object)

\item \(X \rightarrow 0\) is the functor that maps \(X\) into the emtpy set (initial object)
\end{itemize}
\end{frame}

\begin{frame}[label={sec:orge918dca}]{Equivalence between functors}
\begin{itemize}
\item There are isomorphisms that make some functors equivalent, e.g. \(X \times
      Y = Y \times X\) or \((X \times (Y + Z)) = (X \times Y) + (X \times Z)\).

\item They follow the same rules of addition and multiplication of numbers
\end{itemize}
\end{frame}
\begin{frame}[label={sec:org15a9e60}]{Functor algebra}
Given a polynomial (endo-)functor \(T\) that maps \(X \in C\) to \(T(X) \in
    C\), a functor algebra for \(T\) is a pair \((U,a)\):

\begin{itemize}
\item an object \(U \in C\) (\alert{carrier})
\item a morphism \(a: T(U) \rightarrow U\) (\alert{algebra structure})
\end{itemize}

\(a\) must be defined by a cotuple of several functions whose signature is
specified by \(T\). For example, $$T(X) = 1 + X + (X \times X)$$ defines $$e:
    1 \rightarrow U, i: X \rightarrow X, m: (X \times X) \rightarrow X$$ which
might encode the signature of group operations.
\end{frame}

\begin{frame}[label={sec:org194998d}]{Functor algebra example - natural numbers}
\(0\) and \(S\) maps over natural numbers can be seen as a functor algebra
\(([0,S],\mathbb{N})\) of the functor \(T(X) = 1 + X\).
\end{frame}

\begin{frame}[label={sec:org440f597}]{Homomorphisms of algebras}
Given two T-functor algebras \((U, a: T(-) \rightarrow -)\) and \((V, b: T(-) \rightarrow -)\),
a homomorphism of algebras from \((U,a)\) to \((V,b)\) is a function \(f: U \rightarrow V\) which commutes
with the operations $$ f \circ a = b \circ T(f)$$
\end{frame}

\begin{frame}[label={sec:org6f6284c}]{Whole point of (co-)algebras}
\begin{itemize}
\item define functions indirectly exploiting fixed (co-)algebraic constructors
and finality/initiality
\item so one has to specify only the initial algebra \(A\) and another algebra \(B\)
and automatically the morphism can be built. In general, the morphisms
used by \(B\) are not recursive.
\end{itemize}
\end{frame}

\begin{frame}[label={sec:org62c325a}]{Initial algebras}
\begin{itemize}
\item Fixed a functor \(T\), functor algebras \((U, a: T(U) \rightarrow U)\) can be
seen as categorical objects with algebra homomorphisms as morphism.

\item In this category, an \alert{initial algebra} is an algebra for which there exist
one and only one morphism from it to all the other algebras.
\end{itemize}
\end{frame}

\begin{frame}[label={sec:org635e52a}]{Lambek's lemma}
if \(a: T(U) \rightarrow U\) is an initial algebra then there is an
inverse \(a^{-1}: U \rightarrow T(U)\) such that \(T(U) \simeq U\).
\end{frame}

\begin{frame}[label={sec:org48b8cdf}]{An initial algebra defines homomorphisms by induction (integers)}
\begin{itemize}
\item Let's indicate with \((N, [0,S]: 1 + N \rightarrow N\)) the \alert{initial} algebra of
natural numbers.

\item An object \(Q\) equipped with a functor algebra $$(Q, [q_0,q_S]:
      1+Q\rightarrow Q)$$ defines, by induction, one and only one algebra
morphism \(f: N \rightarrow Q\).

\begin{itemize}
\item \(f \circ 0 = q_0\)
\item \(f \circ S = q_S \circ f\)
\end{itemize}

i.e, $$f = n \mapsto q_S^n(q_0)$$
\end{itemize}
\end{frame}

\begin{frame}[label={sec:org45f1971}]{An initial algebra defines homomorphisms by induction (lists)}
\begin{itemize}
\item Let's indicate with \((A^*, [e,c]: 1 + A \times A^* \rightarrow A^*\)) the
\alert{initial} algebra of lists of \(A\).

\item An algebra $$(Q^*, [q_e,q_c]:
      1+A \times Q^* \rightarrow Q^*)$$ defines, by induction, one and only one algebra
morphism \(f: A^* \rightarrow Q^*\).

\begin{itemize}
\item \(f \circ e = q_e\)
\item \(f \circ c = q_c \circ (id \times f)\)
\end{itemize}

\item For example, an algebra $$(\mathbb{N}, [0, S \circ \pi]: 1+A \times
      \mathbb{N} \rightarrow \mathbb{N})$$ defines implicitly the length of a list
\end{itemize}
\end{frame}

\begin{frame}[label={sec:orge21875e}]{Proofs by induction with initial algebras}
Consider that, given an initial algebra \((U, a: T(U) \rightarrow U)\), all
morph. \(f: (U,a) \rightarrow (V,b)\) to the same algebra \((V,b)\) are
equivalent. So,

\begin{itemize}
\item if we want to demonstrate that a function \(\phi: U \rightarrow U\) has
some property and
\item if we can factor it into an algebra homomorphism:
\(prop_1 \circ \phi: U \rightarrow V\) and
\item can find another expression for
the homomorphism \(prop_2: U \rightarrow V\)
\item then they are equivalent, i.e.:
$$prop_1 \circ \phi = prop_2$$
\end{itemize}
\end{frame}

\begin{frame}[label={sec:orgfe10041}]{Principle of list induction}
To show that a predicate/set \(P\) is equal to the set of all lists \(A*\), it is enough to
show the following:

\begin{itemize}
\item \(empty \in P\)
\item \(\forall a, \alpha \in P \Rightarrow cons(a,\alpha) \in P\)
\end{itemize}

i.e., \(P\) can be understood as an algebra $$(U, [empty,cons]: 1 + A\times U
    \rightarrow U)$$ for which we know an initial algebra (the one of lists
\(A^*\)). If this is the case, the inclusion morphism \(i: P \rightarrow A^*\) is
an isomorphism and \(P = A^*\) or \emph{\(P\) holds for all \(A^*\)}.
\end{frame}
\subsection{Co-algebras}
\label{sec:org06428e0}
\begin{frame}[label={sec:org743665f}]{Co-algebra}
Given a polynomial (endo-)functor \(T\) that maps \(X \in C\) to \(T(X) \in
    C\), a functor co-algebra for \(T\) is a pair \$(U=

\begin{itemize}
\item an object \(U \in C\) (\alert{carrier})
\item a morphism \(c: U \rightarrow T(U)\) (\alert{co-algebra structure})
\end{itemize}

\(c\) must be defined by a tuple of several functions whose signature is
specified by \(T\). For example, $$T(X) = A \times X$$ defines $$(value,next):
    (U \rightarrow A, U \rightarrow U)$$ for any process \(U\).
\end{frame}
\begin{frame}[label={sec:orgb1de7b0}]{Homomorphisms of co-algebras}
Given two T-functor co-algebras \((U, a: - \rightarrow T(-))\) and \((V, b: - \rightarrow T(-))\),
a homomorphism of algebras from \((U,a)\) to \((V,b)\) is a function \(f: U \rightarrow V\) which commutes
with the operations $$ T(f) \circ a = b \circ f$$
\end{frame}
\begin{frame}[label={sec:orgf0ae614}]{Final co-algebra}
\begin{itemize}
\item A final coalgebra \(d: W \rightarrow T(W)\) is a coalgebra such that for
every coalgebra \(c: U \rightarrow T(U)\) there exists a unique map of
co-algebras \(f: (U,c) \rightarrow (W,d)\)

\item Note: we have \alert{initial algebras} and \alert{final} \alert{co-algebras}
\item Example: the final algebra of \(T(X) = A \times X\) is \((A^\mathbb{N}, <h,
      t>)\) where \(A^\mathbb{N}\) is the set of infinite sequences over \(A\), while
\(h,t\) are the \alert{head} and \alert{tail} functions.
\end{itemize}
\end{frame}


\subsection{Natural transformations}
\label{sec:org1a180d2}
\begin{frame}[fragile,label={sec:orge322de5}]{Natural transformations and parametric polymorphism}
 \begin{itemize}
\item In PP, I must use a single expression for a parametric function
\item A lot of parametric polymorphic functions \texttt{F a -> G a} are natural
transformations by default (theorems for free) and can be used to optimize
code, e.g.: $$safeTail \circ fmap~f = fmap~f \circ safeTail$$
\item Note, typeclasses represent \emph{ad-hoc} polymorphism.
\item The return of a monad is a natural transformation
\end{itemize}
\end{frame}
\end{document}
